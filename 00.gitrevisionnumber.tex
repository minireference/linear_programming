%   below is some really useful code for automatically inserting the version
%   number below the title of the paper. This way we know which version we
%   have printed out and can see what has changed since then.
%   (uncomment one level to activate -- probably will only work on your Mac)
%
%%================================================================================
%% CALL OUT TO SHELL to get current hg revision number 
\newread\myinput
% We use '\jobname.temp' to create a uniquely-named temporary file
\immediate\write18{git log -l1 | grep "^commit" | cut -c-14 > '\jobname.temp'}
\immediate\openin\myinput=\jobname.temp
% The group localizes the change to \endlinechar
\bgroup
  \endlinechar=-1
  \immediate\read\myinput to \localline
  % Since everything in the group is local, we have to explicitly make the
  % assignment global
  \global\let\gitrevisionnumber\localline
\egroup
\immediate\closein\myinput
% Clean up after ourselves
\immediate\write18{rm -f -- '\jobname.temp'}
%\hgrevisionnumber used to be \myresult
%\dosomething{\myresult}
% via http://stackoverflow.com/questions/2671079/how-can-i-save-shell-output-to-a-variable-in-latex
%%================================================================================

